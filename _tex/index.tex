% Options for packages loaded elsewhere
\PassOptionsToPackage{unicode}{hyperref}
\PassOptionsToPackage{hyphens}{url}
\PassOptionsToPackage{dvipsnames,svgnames,x11names}{xcolor}
%
\documentclass[
  letterpaper,
  DIV=11,
  numbers=noendperiod]{scrartcl}

\usepackage{amsmath,amssymb}
\usepackage{iftex}
\ifPDFTeX
  \usepackage[T1]{fontenc}
  \usepackage[utf8]{inputenc}
  \usepackage{textcomp} % provide euro and other symbols
\else % if luatex or xetex
  \usepackage{unicode-math}
  \defaultfontfeatures{Scale=MatchLowercase}
  \defaultfontfeatures[\rmfamily]{Ligatures=TeX,Scale=1}
\fi
\usepackage{lmodern}
\ifPDFTeX\else  
    % xetex/luatex font selection
\fi
% Use upquote if available, for straight quotes in verbatim environments
\IfFileExists{upquote.sty}{\usepackage{upquote}}{}
\IfFileExists{microtype.sty}{% use microtype if available
  \usepackage[]{microtype}
  \UseMicrotypeSet[protrusion]{basicmath} % disable protrusion for tt fonts
}{}
\makeatletter
\@ifundefined{KOMAClassName}{% if non-KOMA class
  \IfFileExists{parskip.sty}{%
    \usepackage{parskip}
  }{% else
    \setlength{\parindent}{0pt}
    \setlength{\parskip}{6pt plus 2pt minus 1pt}}
}{% if KOMA class
  \KOMAoptions{parskip=half}}
\makeatother
\usepackage{xcolor}
\setlength{\emergencystretch}{3em} % prevent overfull lines
\setcounter{secnumdepth}{5}
% Make \paragraph and \subparagraph free-standing
\makeatletter
\ifx\paragraph\undefined\else
  \let\oldparagraph\paragraph
  \renewcommand{\paragraph}{
    \@ifstar
      \xxxParagraphStar
      \xxxParagraphNoStar
  }
  \newcommand{\xxxParagraphStar}[1]{\oldparagraph*{#1}\mbox{}}
  \newcommand{\xxxParagraphNoStar}[1]{\oldparagraph{#1}\mbox{}}
\fi
\ifx\subparagraph\undefined\else
  \let\oldsubparagraph\subparagraph
  \renewcommand{\subparagraph}{
    \@ifstar
      \xxxSubParagraphStar
      \xxxSubParagraphNoStar
  }
  \newcommand{\xxxSubParagraphStar}[1]{\oldsubparagraph*{#1}\mbox{}}
  \newcommand{\xxxSubParagraphNoStar}[1]{\oldsubparagraph{#1}\mbox{}}
\fi
\makeatother


\providecommand{\tightlist}{%
  \setlength{\itemsep}{0pt}\setlength{\parskip}{0pt}}\usepackage{longtable,booktabs,array}
\usepackage{calc} % for calculating minipage widths
% Correct order of tables after \paragraph or \subparagraph
\usepackage{etoolbox}
\makeatletter
\patchcmd\longtable{\par}{\if@noskipsec\mbox{}\fi\par}{}{}
\makeatother
% Allow footnotes in longtable head/foot
\IfFileExists{footnotehyper.sty}{\usepackage{footnotehyper}}{\usepackage{footnote}}
\makesavenoteenv{longtable}
\usepackage{graphicx}
\makeatletter
\newsavebox\pandoc@box
\newcommand*\pandocbounded[1]{% scales image to fit in text height/width
  \sbox\pandoc@box{#1}%
  \Gscale@div\@tempa{\textheight}{\dimexpr\ht\pandoc@box+\dp\pandoc@box\relax}%
  \Gscale@div\@tempb{\linewidth}{\wd\pandoc@box}%
  \ifdim\@tempb\p@<\@tempa\p@\let\@tempa\@tempb\fi% select the smaller of both
  \ifdim\@tempa\p@<\p@\scalebox{\@tempa}{\usebox\pandoc@box}%
  \else\usebox{\pandoc@box}%
  \fi%
}
% Set default figure placement to htbp
\def\fps@figure{htbp}
\makeatother
% definitions for citeproc citations
\NewDocumentCommand\citeproctext{}{}
\NewDocumentCommand\citeproc{mm}{%
  \begingroup\def\citeproctext{#2}\cite{#1}\endgroup}
\makeatletter
 % allow citations to break across lines
 \let\@cite@ofmt\@firstofone
 % avoid brackets around text for \cite:
 \def\@biblabel#1{}
 \def\@cite#1#2{{#1\if@tempswa , #2\fi}}
\makeatother
\newlength{\cslhangindent}
\setlength{\cslhangindent}{1.5em}
\newlength{\csllabelwidth}
\setlength{\csllabelwidth}{3em}
\newenvironment{CSLReferences}[2] % #1 hanging-indent, #2 entry-spacing
 {\begin{list}{}{%
  \setlength{\itemindent}{0pt}
  \setlength{\leftmargin}{0pt}
  \setlength{\parsep}{0pt}
  % turn on hanging indent if param 1 is 1
  \ifodd #1
   \setlength{\leftmargin}{\cslhangindent}
   \setlength{\itemindent}{-1\cslhangindent}
  \fi
  % set entry spacing
  \setlength{\itemsep}{#2\baselineskip}}}
 {\end{list}}
\usepackage{calc}
\newcommand{\CSLBlock}[1]{\hfill\break\parbox[t]{\linewidth}{\strut\ignorespaces#1\strut}}
\newcommand{\CSLLeftMargin}[1]{\parbox[t]{\csllabelwidth}{\strut#1\strut}}
\newcommand{\CSLRightInline}[1]{\parbox[t]{\linewidth - \csllabelwidth}{\strut#1\strut}}
\newcommand{\CSLIndent}[1]{\hspace{\cslhangindent}#1}

\KOMAoption{captions}{tableheading}
\makeatletter
\@ifpackageloaded{caption}{}{\usepackage{caption}}
\AtBeginDocument{%
\ifdefined\contentsname
  \renewcommand*\contentsname{Table of contents}
\else
  \newcommand\contentsname{Table of contents}
\fi
\ifdefined\listfigurename
  \renewcommand*\listfigurename{List of Figures}
\else
  \newcommand\listfigurename{List of Figures}
\fi
\ifdefined\listtablename
  \renewcommand*\listtablename{List of Tables}
\else
  \newcommand\listtablename{List of Tables}
\fi
\ifdefined\figurename
  \renewcommand*\figurename{Figure}
\else
  \newcommand\figurename{Figure}
\fi
\ifdefined\tablename
  \renewcommand*\tablename{Table}
\else
  \newcommand\tablename{Table}
\fi
}
\@ifpackageloaded{float}{}{\usepackage{float}}
\floatstyle{ruled}
\@ifundefined{c@chapter}{\newfloat{codelisting}{h}{lop}}{\newfloat{codelisting}{h}{lop}[chapter]}
\floatname{codelisting}{Listing}
\newcommand*\listoflistings{\listof{codelisting}{List of Listings}}
\makeatother
\makeatletter
\makeatother
\makeatletter
\@ifpackageloaded{caption}{}{\usepackage{caption}}
\@ifpackageloaded{subcaption}{}{\usepackage{subcaption}}
\makeatother

\usepackage{bookmark}

\IfFileExists{xurl.sty}{\usepackage{xurl}}{} % add URL line breaks if available
\urlstyle{same} % disable monospaced font for URLs
\hypersetup{
  pdftitle={La Palma Earthquakes},
  pdfauthor={Steve Purves; Rowan Cockett},
  pdfkeywords={La Palma, Earthquakes},
  colorlinks=true,
  linkcolor={blue},
  filecolor={Maroon},
  citecolor={Blue},
  urlcolor={Blue},
  pdfcreator={LaTeX via pandoc}}


\title{La Palma Earthquakes}
\author{Steve Purves \and Rowan Cockett}
\date{2025-02-05}

\begin{document}
\maketitle
\begin{abstract}
In September 2021, a significant jump in seismic activity on the island
of La Palma (Canary Islands, Spain) signaled the start of a volcanic
crisis that still continues at the time of writing. Earthquake data is
continually collected and published by the Instituto Geográphico
Nacional (IGN). \ldots{}
\end{abstract}


\section{Introduction}\label{introduction}

\begin{quote}
Introduction and Motivation: What is the problem you are trying to
solve? Why is this problem interesting? What has been tried before? What
have been the shortcoming of those approaches that necessitate your
efforts? Often, this section will conclude with a subsection (or
paragraph) outlining ``our contributions.'' What is the new knowledge
that this paper contributes?
\end{quote}

\textsubscript{Source:
\href{https://sds-capstone.github.io/capstone-repo/index.qmd.html}{Article
Notebook}}

\phantomsection\label{cell-fig-timeline}
\begin{figure}[H]

\centering{

\pandocbounded{\includegraphics[keepaspectratio]{index_files/figure-pdf/fig-timeline-1.pdf}}

}

\caption{\label{fig-timeline}Timeline of recent earthquakes on La Palma}

\end{figure}%

\textsubscript{Source:
\href{https://sds-capstone.github.io/capstone-repo/index.qmd.html}{Article
Notebook}}

\textsubscript{Source:
\href{https://sds-capstone.github.io/capstone-repo/index.qmd.html}{Article
Notebook}}

Based on data up to and including 1971, eruptions on La Palma happen
every 79.8 years on average.

\begin{figure}

\centering{

\pandocbounded{\includegraphics[keepaspectratio]{images/la-palma-map.png}}

}

\caption{\label{fig-map}Map of La Palma}

\end{figure}%

La Palma is one of the west most islands in the Volcanic Archipelago of
the Canary Islands (Figure~\ref{fig-map}).

\subsection{Relevant work}\label{relevant-work}

Studies of the magma systems feeding the volcano, such as Marrero et al.
(2019), have proposed that there are two main magma reservoirs feeding
the Cumbre Vieja volcano; one in the mantle (30-40km depth) which
charges and in turn feeds a shallower crustal reservoir (10-20km depth).

Eight eruptions have been recorded since the late 1400s
(Figure~\ref{fig-timeline}).

\subsection{Our contribution}\label{our-contribution}

Data and methods are discussed Section~\ref{sec-data} and
Section~\ref{sec-methods}, respectively.

\section{Data}\label{sec-data}

\begin{quote}
Data: Where did it come from? What are some basic summary statistics,
variable definitions, and/or visualizations that help the reader
understand the data you are working with?
\end{quote}

Read a clean version of data:

\textsubscript{Source:
\href{https://sds-capstone.github.io/capstone-repo/index.qmd.html}{Article
Notebook}}

Create spatial plot:

\phantomsection\label{cell-fig-spatial-plot}
\begin{figure}[H]

\centering{

\pandocbounded{\includegraphics[keepaspectratio]{index_files/figure-pdf/fig-spatial-plot-1.pdf}}

}

\caption{\label{fig-spatial-plot}Locations of earthquakes on La Palma
since 2017}

\end{figure}%

\textsubscript{Source:
\href{https://sds-capstone.github.io/capstone-repo/index.qmd.html}{Article
Notebook}}

Figure~\ref{fig-spatial-plot} shows the location of recent Earthquakes
on La Palma.

\section{Methods}\label{sec-methods}

\begin{quote}
Methods: What did you actually do? What techniques or methods did you
employ? What were the specifications for any statistical models you
used? What software or packages did you use or develop?
\end{quote}

Let \(x\) denote the number of eruptions in a year. Then, \(x\) can be
modeled by a Poisson distribution

\begin{equation}\phantomsection\label{eq-poisson}{
p(x) = \frac{e^{-\lambda} \lambda^{x}}{x !}
}\end{equation}

where \(\lambda\) is the rate of eruptions per year. Using
Equation~\ref{eq-poisson}, the probability of an eruption in the next
\(t\) years can be calculated.

\begin{longtable}[]{@{}ll@{}}
\caption{Recent historic eruptions on La
Palma}\label{tbl-history}\tabularnewline
\toprule\noalign{}
Name & Year \\
\midrule\noalign{}
\endfirsthead
\toprule\noalign{}
Name & Year \\
\midrule\noalign{}
\endhead
\bottomrule\noalign{}
\endlastfoot
Current & 2021 \\
Teneguía & 1971 \\
Nambroque & 1949 \\
El Charco & 1712 \\
Volcán San Antonio & 1677 \\
Volcán San Martin & 1646 \\
Tajuya near El Paso & 1585 \\
Montaña Quemada & 1492 \\
\end{longtable}

Table~\ref{tbl-history} summarises the eruptions recorded since the
colonization of the islands by Europeans in the late 1400s.

\section{Results}\label{results}

\begin{quote}
Results: What did you learn about the problem you identified in Section
1? This is where you put the tables, figures, and analytics by-products
of your work.
\end{quote}

\section{Conclusion}\label{conclusion}

\begin{quote}
Conclusion: What are the limitations of your work? What are some next
steps that someone (either you or another research group) should
consider in attempting to further your work? Remind us one last time
about what you did.
\end{quote}

\subsection{Limitations}\label{limitations}

\subsection{Future work}\label{future-work}

\subsection{Final thoughts}\label{final-thoughts}

\section{Ethical statement}\label{ethical-statement}

\begin{quote}
Drawing on what you have learned about data science ethics in this
class, discuss any ethical considerations in your project. For some
projects, this statement could be quite short (one paragraph may
suffice). For other projects, more detail may be needed (no more than 2
pages).

Be expansive and creative in your thinking about \textbf{possible}
ethical considerations. One way to do this assignment poorly would be to
write a short statement asserting that there are no ethical
considerations, only to have me think of several fairly obvious ones.
\end{quote}

\section{Acknowledgements}\label{acknowledgements}

\begin{quote}
Were there any people or organizations who helped you that you would
like to acknowledge?
\end{quote}

\section*{References}\label{references}
\addcontentsline{toc}{section}{References}

\begin{quote}
References: Every reference (except for books) needs a DOI (strongly
preferred) or URL. Use \href{https://scholar.google.com}{Google Scholar}
to help with BibTeX. Every book needs publisher location (e.g.~``City,
ST: Publisher''). 🙄 See
\href{https://smithcollege-sds.github.io/sds100/lab_12_bibliographies.html}{SDS
100: Bibliographies}
\end{quote}

\phantomsection\label{refs}
\begin{CSLReferences}{1}{0}
\bibitem[\citeproctext]{ref-marrero2019}
Marrero, José, Alicia García, Manuel Berrocoso, Ángeles Llinares,
Antonio Rodríguez-Losada, and R. Ortiz. 2019. {``Strategies for the
Development of Volcanic Hazard Maps in Monogenetic Volcanic Fields: The
Example of {La} {Palma} ({Canary} {Islands}).''} \emph{Journal of
Applied Volcanology} 8 (July).
\url{https://doi.org/10.1186/s13617-019-0085-5}.

\end{CSLReferences}




\end{document}
